\subsection{Average Data Usage per Users over Hours on each Month}

In this part, we will compare the average download data per user over hour and the average upload data per user over hour on each month before and after COVID-19. The download data is the number of bytes that the customer received from the internet to wired LAN devices and the Wi-Fi devices, and the upload data is the number of bytes that the customer transmitted from wired LAN devices and Wi-Fi devices to the internet. 

Since the recorded time in data set is finished in coordinated universal time, in order to get the time of download or upload event at local time, we add the time zone offset from the user profile to the recorded time.  We extract the year, month, and hour from the local time, and handle some special cases: (1) for hour, there is some location behind coordinated universal time and if the hour in local time is negative, add the value by 24. (2) For month, if the value of hour in local time is negative and the day is 1, we subtract the value of extracted month by 1 because it will be the last date of the previous month, and (3) for year, if the value of hour in local time is negative, the month is 1 and the day is 1, we subtract the value of extracted year by 1 since it will be the last date of the previous year. For the users, we focus on users who are present throughout the course of the study. We notice that the number of bytes that the customer transmitted from wired LAN devices to the internet for some users is the value of maximum integer. The possible reason is that the device measuring that value may be broken so we ignore all the users who have those values.

For each month, the average number of download data and that of upload data are defined as 

\begin{equation}
    AVG_{\text{download}} =\frac{SUM(\text{download data})}{COUNT(\text{persistent users})}
\end{equation}
\begin{equation}
    AVG_{\text{upload}} =\frac{SUM(\text{upload data})}{COUNT(\text{persistent users})}
\end{equation}
where ``persistent users" means the users who are present throughout the course of the study.